% ---Notation macros---

% math
\let\oldvec\vec
% \renewcommand{\vec}[1]{\bm{\mathbf{#1}}}
\renewcommand{\vec}[1]{\bm{#1}} % \bm vs \mathbf
\newcommand{\vecf}[1]{\bm{#1}} % \bm vs \mathbf
\newcommand{\vecelem}[1]{#1}
\newcommand{\vecarrow}[1]{\vv{#1}}
\newcommand{\mat}[1]{\bm{#1}}
\newcommand{\matelem}[1]{\mathit{#1}}
\newcommand{\ten}[1]{\mathbf{#1}}
\newcommand{\tenelem}[1]{\mathrm{#1}}
\newcommand{\set}[1]{\mathbb{#1}}

\newcommand{\trans}[1]{#1^{\mathsf{T}}}
\newcommand{\ifrac}[2]{\sfrac{#1}{#2}}
\newcommand{\deriv}[2]{\frac{\text{d}#1}{\text{d}#2}}
\newcommand{\partderiv}[2]{\frac{\partial #1}{\partial #2}}
\newcommand{\ideriv}[2]{\ifrac{\text{d}#1}{\text{d}#2}}
\newcommand{\ipartderiv}[2]{\ifrac{\partial #1}{\partial #2}}

\DeclarePairedDelimiter{\norm}{\lVert}{\rVert}

% machine learning
\newcommand{\param}{\lambda}


% misc
\newcommand{\keyword}[1]{\textbf{#1}}
\newcommand{\para}{\par\medskip}
\newcommand{\ds}{\displaystyle}

\newcommand{\keycode}[1]{\texttt{#1}}
\newcommand{\code}[1]{\texttt{#1}}


\let\oldcite\cite
\renewcommand{\cite}[1]{{\color{red}(\oldcite{#1})}\\}



% --- TIKZ ---

\NewEnviron{defbox}[1]
{
  \centering

  \tikzstyle{mybox} = [draw=red, fill=blue!40!pagecolor, very thick, rectangle, rounded corners, inner sep=10pt, inner ysep=20pt]
  \tikzstyle{fancytitle} = [fill=red, text=white]
  \begin{tikzpicture}
    \node [mybox] (box) {%
      \begin{minipage}{1.0\textwidth}
        \BODY
      \end{minipage}
    };
    \node [right,inner xsep=1em,fill=red!75, text=white,outer sep=0pt,text height=2ex,text depth=.5ex] (title)
    at ([shift={(-1em,0pt)}]box.north west) {#1};
    \fill[red!50!black] (title.north east) -- +(-1em,1em) -- +(-1em,0) -- cycle;
    \fill[red!50!black] (title.south west) -- +(1em,-1em) -- +(1em,0) -- cycle;

  \end{tikzpicture}
}


\NewEnviron{infobox}[1]
{
  \centering

  \tikzstyle{mybox} = [draw=blue, fill=green!40!pagecolor, very thick, rectangle, rounded corners, inner sep=10pt, inner ysep=20pt]
  \tikzstyle{fancytitle} = [fill=blue, text=white]
  \begin{tikzpicture}
    \node [mybox] (box) {%
      \begin{minipage}{1.0\textwidth}
        \BODY
      \end{minipage}
    };
    \node [right,inner xsep=1em,fill=blue!75, text=white,outer sep=0pt,text height=2ex,text depth=.5ex] (title)
    at ([shift={(-1em,0pt)}]box.north west) {#1};
    \fill[blue!50!black] (title.north east) -- +(-1em,1em) -- +(-1em,0) -- cycle;
    \fill[blue!50!black] (title.south west) -- +(1em,-1em) -- +(1em,0) -- cycle;

  \end{tikzpicture}
}

% \newcommand{\code}{\mint[frame=lines,framesep=2mm,baselinestretch=1.2,bgcolor=lightgray,fontsize=\footnotesize,linenos]{python}}

% \newenvironment{codeblock}
% {
%   \begin{minted}[frame=lines,framesep=2mm,baselinestretch=1.2,bgcolor=lightgray,fontsize=\footnotesize,linenos]{python}
% }
% {
% \end{minted}
% }

% conditional compilation
\newif\ifcp
\cptrue % if commented out -> no compilation

% \ifcp
   %% CONDITIONAL CODE
% \fi


%%% Local Variables:
%%% mode: latex
%%% TeX-master: "../main"
%%% End:
