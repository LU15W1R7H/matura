\chapter*{Schlusswort}
\addcontentsline{toc}{chapter}{Schlusswort}

\section*{Ausblick}
\subsection*{Weitere Architekturen}
Variational Autencoder
Generative Adversial Networks

Recurrent Neural Networks

\subsection*{Weitere Frameworks}
TensorFlow 2.0
PyTorch



\section*{Persoenlicher Eindruck}

% ---------------------------------

\begin{appendices}
\chapter{Programmcode}

\begin{minted}[frame=lines,framesep=2mm,baselinestretch=1.2,bgcolor=lightgray,fontsize=\footnotesize,linenos]{python}
import numpy as np
import matplotlib.pyplot as plt

import tensorflow as tf

(x_train, _), (x_test, _) = tf.keras.datasets.mnist.load_data()
x_train = x_train.astype('float32') / 255.
x_test = np.reshape(x_test, (len(x_test),28,28,1))
x_test = x_test.astype('float32') / 255.
x_train = np.reshape(x_train, (len(x_train),28,28,1))

noise_factor = 0.5
x_train_noisy = x_train + noise_factor * np.random.normal(loc=0.0, scale=1.0, size=x_train.shape)
x_test_noisy = x_test + noise_factor * np.random.normal(loc=0.0, scale=1.0, size=x_test.shape)
x_train_noisy = np.clip(x_train_noisy, 0., 1.)
x_test_noisy = np.clip(x_test_noisy, 0., 1.)


input_data = tf.keras.Input(shape=(28,28,1))

econv0 = tf.keras.layers.Conv2D(filters=32, kernel_size=(3,3), strides=(1,1),padding='same',activation='relu')(input_data)
emaxpool0 = tf.keras.layers.MaxPooling2D(pool_size=(2,2),strides=None,padding='same')(econv0)
econv1 = tf.keras.layers.Conv2D(filters=32, kernel_size=(3,3), strides=(1,1),padding='same',activation='relu')(emaxpool0)
emaxpool1 = tf.keras.layers.MaxPooling2D(pool_size=(2,2), strides=None,padding='same')(econv1)
encoded =  emaxpool1 # bottleneck (shape=(7,7,32))
dconv0 = tf.keras.layers.Conv2D(filters=32, kernel_size=(3,3), strides=(1,1), padding='same', activation='relu')(encoded)
dupsample0 = tf.keras.layers.UpSampling2D(size=(2,2), interpolation='nearest')(dconv0)
dconv1 = tf.keras.layers.Conv2D(filters=32, kernel_size=(3,3), strides=(1,1), padding='same', activation='relu')(dupsample0)
dupsample1 = tf.keras.layers.UpSampling2D(size=(2,2), interpolation='nearest')(dconv1)
dconv2 = tf.keras.layers.Conv2D(filters=1,kernel_size=(3,3), strides=(1,1), padding='same', activation='sigmoid')(dupsample1)
decoded = dconv2

autoencoder = tf.keras.Model(input_data, decoded)
autoencoder.compile(optimizer='adadelta',loss='binary_crossentropy')

print(autoencoder.summary())


autoencoder.fit(x=x_train_noisy, y=x_train, batch_size=128, epochs=100, shuffle=True,validation_data=(x_test_noisy,x_test),callbacks=[tf.keras.callbacks.TensorBoard(log_dir='/tmp/denoiser',histogram_freq=0,write_graph=False)])

decoded_imgs = autoencoder.predict(x_test)

autoencoder.save('denoiser.model')

n = 10
plt.figure(figsize=(20,4))
for i in range(n):
    # original
    ax = plt.subplot(2, n, i+1)
    plt.imshow(x_test_noisy[i].reshape(28,28))
    plt.gray()
    ax.get_xaxis().set_visible(False)
    ax.get_yaxis().set_visible(False)

    # reconstruction
    ax = plt.subplot(2,n,i+1+n)
    plt.imshow(decoded_imgs[i].reshape(28,28))
    plt.gray()
    ax.get_xaxis().set_visible(False)
    ax.get_yaxis().set_visible(False)
plt.show()
\end{minted}

\chapter{Ausfuehrungen zu TensorFlow}
Backend

\chapter{Intuitiver Beweis fuer UAT?}

\end{appendices}

\printbibliography[heading=bibintoc]
\pagebreak

\addcontentsline{toc}{chapter}{Abbildungsverzeichniss}
\listoffigures
\pagebreak

\addcontentsline{toc}{chapter}{Tabellenverzeichniss}
\listoftables
\pagebreak

\addcontentsline{toc}{chapter}{Selbstständigkeitserklärung}
\chapter*{Selbstständigkeitserklärung}
Ich erkläre hiermit, dass ich diese Arbeit selbständig durchgeführt und keine anderen als die angegebene Quellen, Hilfsmittel und Hilfspersonen beigezogen habe. Alle Textstellen in der Arbeit, die wörtlich oder sinngemäss aus Quellen entnommen wurden, habe ich als solche gekennzeichnet.

\vspace{2cm}
\begin{center}
  \noindent\rule{5cm}{0.4pt}\\
  Luis Wirth
\end{center}


%%% Local Variables:
%%% mode: latex
%%% TeX-master: "../main"
%%% End:
