\backgroundsetup{
  scale=1,
  opacity=0.6,
  angle=0,
  % contents={\includegraphics[width=\paperwidth]{images/bild.jpg}}
}

\begin{titlepage}
  \centering

  {\scshape\LARGE Gymnasium Oberwil\par}
  \vspace{1cm}
  {\scshape\Large Maturaarbeit\par}
  \vspace{1.5cm}
  {\huge\bfseries Maschinelles Lernen mit TensorFlow\par}
  \vspace{0.1cm}
  {\large Entwicklung eines Convolutional Denoising Autoencoders\par}
  \vspace{2cm}
  {\Large\itshape Luis Wirth, 4a\par}
  \vfill
  Betreut von\par
  Dr. Jonas \textsc{Gloor}

  \vfill
  {\large Abgegeben im September 2019\par}
\end{titlepage}


\addcontentsline{toc}{chapter}{Inhaltsverzeichnis}
\tableofcontents
\pagebreak

\addcontentsline{toc}{chapter}{Vorwort}
\chapter*{Vorwort}

Sollte insgesamt eine Seite umfassen.

Auf die Leitfragenaenderung eingehen. (Weg von der alten zur neunen Frage)

\addcontentsline{toc}{chapter}{Einleitung}
\chapter*{Einleitung}
ALLGEMEINES ZUM THEMENGEBIET MACHINELLES LERNEN

INFORMATIONEN ZU DIESER ARBEIT: INHALT AUFBAU ETC
Das Ziel dieser Arbeit ist es, dem Leser ein umfangreiches Grundverstaentniss
ueber Maschinelles Lernen zu bieten. Dieses umfasst die theoretischen Grundlagen
zu modellbasiertem Lernen, Kuenstlichen Neuronalen Netzen, Convolutional Neural
Networks und Autoencodern. Mithilfe dieser erarbeiten Theorie wird dann noch die
die praktische Umsetzung eines Modells mithilfe von TensorFlow und Keras
erklaert. Bei TensorFlow und Keras handelt es sich um Frameworks fuer
Maschinelles Lernen mit Python. Sie sind in der Industrie sehr verbreitet und
wurden deshalb fuer diese Arbeit gewaehlt.
\para{}
Somit wurde die Leitfrage folgendermassen formuliert:
\textbf{Wie funktioniert Maschinelles Lernen und wie entwickelt man mit diesem
  Wissen ein konkretes Modell in TensorFlow?}
\para{}

Wir werden folgenden groben Aufbau verfolgen:

\begin{enumerate}
  \item{Wir erarbeiten die allgemeine Theorie zum Maschinellen Lernen.}
  \item{Wir betrachten Kuenstliche Neuronale Netze als Modelle fuer ML.}
  \item{Wir betrachten Convolutional Neural Networks und Autoencoder als
      konkrete KNN Architekturen.}
  \item{Wir betrachten die ML-Frameworks TensorFlow und Keras bezueglich ihrer Funktionsweise.}
  \item{Wir programmieren mithilfe der erarbeiten Theorie ein konkretes Modell
      in TensorFlow / Keras.}
    \item{Wir untersuchen das programmiere Modell.}

\end{enumerate}

Fuer das konkrete Modell werden wir einen sogennante
Convolutional-Denoising-Autoencoder entwickeln. In den Theorie Teilen werden wir
lernen wie dieser Funktioniert. Der Grund fuer die Wahl dieses kokreten Modells,
besteht darin, dass es sich dabei um ein Convolutional Neural Network handelt,
welche in den letzten Jahren sehr populaer geworden sind und deshalb interessant
zu betrachten sind. Wir werden dieses Konzept eines Convolutional Neural Network
mit einem Autoencoder koppeln, um einen Convolutional-Denoising-Autoencoder zu
erhalten. Mithilfe von ihm koennen wir Bildrauschen von Bildern wegrechnen lassen.


\addcontentsline{toc}{chapter}{Konvention und Notation}
\chapter*{Konvention und Notation}

\subsection*{Beschriftung}

\begin{center}\textbf{Zahlen und Tensoren}\end{center}
\begin{tabular}{cl}
  $a$ & ein Skalar (Zahl) \\
  $\vec{a}$ & ein Vektor \\
  $\vec{a} \in \set{R}^{n}$ & ein Vektor mit $n$ Komponenten \\
  $\mat{A}$ & eine Matrix \\
  $\mat{A} \in \set{R}^{m \times n}$ & eine Matrix mit $m$ Zeilen und $n$ Spalten \\
  $\ten{A}$ & ein Tensor \\
  $\mathfrak{A,B,C,D,E,\cdots}$ & a \\
  $\mathfrak{a,b,c,d,e,f,g,\cdots}$ & a \\


\end{tabular}

\begin{center}\textbf{Mengen}\end{center}
\begin{tabular}{cl}
  $(a,b)$ & ein geordnetes Paar (2-Tupel) der Elemente $a$ und $b$ \\
  $\set{A}$ & eine Menge \\
  $\set{R}$ & die Menge aller reellen Zahlen \\
  $\set{Z}$ & die Menge aller ganzen Zahlen \\
  $\set{N}$ & die Menge aller natuerlichen Zahlen \\
  % $\set{E}$ & die Menge aller gerader (engl.:\ even) Zahlen \\
  % $\set{O}$ & die Menge aller ungerader (engl.:\ odd) Zahlen \\
  $\{a,b\}$ & die Menge, welche aus den Elementen $a$ und $b$ besteht \\
  $\{1,\ldots,n\}$ & die Menge aller ganzen Zahlen von 1 bis $n$ \\
  $a \in \set{A}$ & $a$ ist ein Element der Menge $\set{A}$ \\

\end{tabular}

\begin{center}\textbf{Indexierung}\end{center}
\begin{tabular}{cl}
  $\vecelem{a}_i$ & die $i$-te Vektorkomponente mit Startindex 1 \\
  ${(\mat{A})}_{i,j}$ & das Matrixelement in der $i$-ten Zeile und der $j$-ten Spalte \\
  $\matelem{A}_{i,j}$ & das Matrixelement in der $i$-ten Zeile und der $j$-ten Spalte \\
  $\mat{A}_{i,:}$ & die Zeile $i$ der Matrix $\mat{A}$ \\
  $\mat{A}_{:,i}$ & die Spalte $i$ der Matrix $\mat{A}$\ \\
  $\ten{A}_{i,:,:}$ & der $i$-te Querschnitt entlang der Hoehe des Tensors $\ten{A}$ \\
  $\ten{A}_{:,i,:}$ & der $i$-te Querschnitt entlang der Breite des Tensors $\ten{A}$ \\
  $\ten{A}_{:,:,i}$ & der $i$-te Querschnitt entlang der Tiefe des Tensors $\ten{A}$ \\
  $\mat{A} = \begin{pmatrix} \end{pmatrix}$ & Definition einer Matrix $\mat{A}$ \\
  $\ten{A} = \begin{bmatrix} \mat{A}_{:,:,1} & \cdots & \mat{A}_{:,:,c} \end{bmatrix}$ & Definiton eines Tensors $\ten{A}$ \\

\end{tabular}

\begin{center}\textbf{Lineare Algebra Operationen}\end{center}
\begin{tabular}{cl}
  $\vec{v} \cdot \vec{w}$ & das Skalarprodukt von $\vec{v}$ mit $\vec{w}$ \\
  $\trans{\mat{A}}$ & das Transponierte einer Matrix $\mat{A}$ \\
  $\mat{A} \odot \mat{B}$ & das elementeweise (Hadamard) Produkt \\
  $\mat{A} * \mat{B}$ & die diskrete Faltung von $\mat{A}$ ueber $\mat{B}$

\end{tabular}

\begin{center}\textbf{Infinitesimalrechnung}\end{center}
\begin{tabular}{cl}
  $f'(x)$ & die Ableitung der Funktion $f$ bezueglich seinem Argument $x$ \\
  $\ds\deriv{y}{x}$ & die Ableitung von $y$ bezueglich $x$ \\[2ex]
  $\ds\partderiv{y}{x}$ & die partielle Ableitung von $y$ bezueglich $x$ \\[2ex]
  $\vecf{\nabla} y$ & der Gradient von $y$\\
  $\vecf{\nabla}_{\vec{x}}y$ & der Gradient von $y$ bezueglich $\vec{x}$ (Vektor) \\
  $\ds\int f(x)\,\text{d}x$ & das unbestimmte Integral der Funktion $f$ bezueglich $x$ \\
  $\ds\int_a^b f(x)\,\text{d}x$ & das bestimmte Integral der Funktion $f$ bezueglich $x$ von $a$ bis $b$ \\

\end{tabular}

\begin{center}\textbf{Funktionen}\end{center}
\begin{tabular}{cl}
  $f: \set{A} \to \set{B}$ & eine Funktion $f$ mit Definitionsmenge $\set{A}$ und Zielmenge $\set{B}$ \\
  $f(x)$ & eine Funktion $f$ mit Argument $x$ (Skalar) \\
  $f(\vec{v})$ & eine Funktion $f$ mit Argument $\vec{v}$ (Vektor) \\
  $\vecf{f}[\vec{v}]$ & die vektorisierte Funktion $\vecf{f}$ mit Argument $\vec{v}$ (Vektor) \\
  $\mathcal{A}$ & ein Funktionenraum \\
  $f * g$ & die Faltung von $f$ ueber $g$ \\

\end{tabular}

\begin{center}\textbf{Statistik}\end{center}
\begin{tabular}{cl}
  $\mathcal{N}$ & die Gauss'sche Normalverteilung \\
  $\phi$ & die Gauss'sche Dichtefunktion \\
  $\mu$ & der Erwartungswert/Mittelwert \\
  $\sigma^2$ & die Varianz
\end{tabular}

Eraehnung von Notation $\set{R}$ fuer Skalare und $\set{R}^n$ fuer Vektoren/Vektorraueme...

%%% Local Variables:
%%% mode: latex
%%% TeX-master: "../main"
%%% End:
