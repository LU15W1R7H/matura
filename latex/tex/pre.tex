\backgroundsetup{
  scale=1,
  opacity=0.6,
  angle=0,
  % contents={\includegraphics[width=\paperwidth]{images/bild.jpg}}
}

\begin{titlepage}
  \centering

  {\scshape\LARGE Gymnasium Oberwil\par}
  \vspace{1cm}
  {\scshape\Large Maturarbeit\par}
  \vspace{1.5cm}
  {\huge\bfseries Bildsynthese von menschlichen Gesichtern mithilfe von Kuenstlicher Intelligenz\par}
  \vspace{0.2cm}
  {\large Eine Arbeit ueber Kuenstliche Neuronale Netze, Convolutional Neural
    Networks, Autoencoders, Generative Modelle und Tensorflow/Keras \par}
  \vspace{2cm}
  {\Large\itshape Luis Wirth, 4a\par}
  \vfill
  Betreut von\par
  Dr. Jonas \textsc{Gloor}

  \vfill
  {\large Abgegeben im September 2019\par}
  \LaTeX{}
\end{titlepage}


\addcontentsline{toc}{chapter}{Inhaltsverzeichniss}
\tableofcontents
\pagebreak

\chapter*{Vorwort}
\addcontentsline{toc}{chapter}{Vorwort}
\pagebreak

\chapter*{Einleitung}
\addcontentsline{toc}{chapter}{Einleitung}

Die Leitfrage lautet: \textbf{Wie können mithilfe von Machine Learning Bilder von künstlichen meschlichen Gesichtern generiert werden?}
\pagebreak

\chapter*{Konvention und Notation}
\addcontentsline{toc}{chapter}{Konvention und Notation}

\subsection*{Beschriftung}

\begin{center}\textbf{Zahlen und Tensoren}\end{center}
\begin{tabular}{cl}
  $a$ & ein Skalar (Zahl) \\
  $\vec{a}$ & ein Vektor \\
  $\vec{a} \in \set{R}^{n}$ & Vektor mit $n$ Komponenten \\
  $\mat{A}$ & eine Matrix \\
  $\mat{A} \in \set{R}^{n \times m}$ & Matrix mit $n$ Zeilen und $m$ Spalten \\
  $\ten{A}$ & ein Tensor \\
  $\mathfrak{A,B,C,D,E,\cdots}$ & a \\
  $\mathfrak{a,b,c,d,e,f,g,\cdots}$ & a \\


\end{tabular}

\begin{center}\textbf{Mengen}\end{center}
\begin{tabular}{cl}
  $(a,b)$ & ein geordnetes Paar (2-Tupel) der Elemente $a$ und $b$ \\
  $\set{A}$ & eine Menge \\
  $\set{R}$ & die Menge aller Reeller Zahlen \\
  $\set{Z}$ & die Menge aller ganzer Zahlen \\
  $\set{N}$ & die Menge aller natuerlicher Zahlen \\
  $\set{E}$ & die Menge aller gerader (engl.:\ even) Zahlen \\
  $\set{O}$ & die Menge aller ungerader (engl.:\ odd) Zahlen \\
  $\{a,b\}$ & die Menge, welche aus Elementen $a$ und $b$ besteht \\
  $\{1,\ldots,n\}$ & Menge aller ganzen Zahlen von 1 bis $n$ \\
  $\in$ & ... ist Element von ...

\end{tabular}

\begin{center}\textbf{Indexierung}\end{center}
\begin{tabular}{cl}
  $\vecelem{a}_i$ & $i$-te Vektorkomponente mit Startindex 1 \\
  ${(\mat{A})}_{i,j} = $ & Matrixelement in der $i$-ten Zeile und der $j$-ten Spalte \\
  $\matelem{A}_{i,j}$ & Matrixelement in der $i$-ten Zeile und der $j$-ten Spalte \\
  $\mat{A}_{i,:}$ & Zeile $i$ der Matrix $\mat{A}$ \\
  $\mat{A}_{:,i}$ & Spalte $i$ der Matrix $\mat{A}$\ \\

\end{tabular}

\begin{center}\textbf{Lineare Algebra Operationen}\end{center}
\begin{tabular}{cl}
  $\vec{v} \cdot \vec{w}$ & Skalarprodukt von $\vec{v}$ mit $\vec{w}$ \\
  $\trans{\mat{A}}$ & das Transponierte einer Matrix $\mat{A}$ \\
  $\mat{A} \odot \mat{B}$ & Elementeweises (Hadamard) Produkt \\
  $\mat{A} * \mat{B}$ & diskrete Faltung von $\mat{A}$ ueber $\mat{B}$

\end{tabular}

\begin{center}\textbf{Infinitesimalrechnung/Differentialrechnung}\end{center}
\begin{tabular}{cl}
  $f'(x)$ & Ableitung der Funktion $f$ bezueglich seinem Argument $x$ \\
  $\ds\deriv{y}{x}$ & Ableitung von $y$ bezueglich $x$ \\[2ex]
  $\ds\partderiv{y}{x}$ & Partielle Ableitung von $y$ bezueglich $x$ \\[2ex]
  $\vecf{\nabla} y$ & Gradient von $y$\\
  $\vecf{\nabla}_{\vec{x}}y$ & Gradient von $y$ bezueglich $\vec{x}$ (Vektor) \\

\end{tabular}

\begin{center}\textbf{Funktionen}\end{center}
\begin{tabular}{cl}
  $f: \set{A} \to \set{B}$ & Funktion $f$ mit Definitionsmenge $\set{A}$ und Zielmenge $\set{B}$ \\
  $f(x)$ & Funktion $f$ mit Argument $x$ (Skalar) \\
  $f(\vec{v})$ & Funktion $f$ mit Argument $\vec{v}$ (Vektor) \\
  $\vecf{f}[\vec{v}]$ & Vektorisierte Funktion $\vecf{f}$ mit Argument $\vec{v}$ (Vektor) \\
  $\mathcal{A}$ & ein Funktionenraum \\
  $f * g$ & Faltung von $f$ ueber g \\

\end{tabular}

\begin{center}\textbf{Statistik}\end{center}
\begin{tabular}{cl}
  $\mathcal{N}$ & Gauss'sche Normalverteilung \\
  $\phi$ & Gauss'sche Dichtefunktion \\
  $\mu$ & Erwartungswert/Mittelwert \\
  $\sigma^2$ & Varianz
\end{tabular}

\begin{center}\textbf{Sonstiges}\end{center}
\begin{tabular}{cl}
  $\forall$ & ``fuer jedes ...'' \\
  $\therefore$ & ``... daraus folgt, dass ...'' \\
  $\exists$ & ``... existiert'' \\
\end{tabular}

\pagebreak

%%% Local Variables:
%%% mode: latex
%%% TeX-master: "../main"
%%% End:
