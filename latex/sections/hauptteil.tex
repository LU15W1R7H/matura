\documentclass[../main]{subfiles}

\begin{document}


Der Themenbereich des Machinellen Lernens beschäftigt sich mit Algorithmen und Modellen, welche mithilfe von Trainingsdaten eine Korrelation zwischen Input und Output erlernt. Dabei werden keine expliziten Instruktionen ausgefuehrt, sondern das Modell erlernt von selbst den Zusammenhang. Unter korrekten Bedingungen (keine Ueberanpassung) findet kein auswenig Lernen statt, sondern ein Generalisieren des Zusammenhangs, anhand von Mustern und Gesetzmassigkeiten.
Diese Modelle machen laufend vorherrsagen ueber den vermuteten Output, welcher mit dem erwarteten Output vergliechen wird und so bewertet wird. Anhand dieser Bewertung wird eine Optimierung des Algorithmus vorgenommen. Somit gewinnt er Verstantniss aus Erfahrung, aehnlich wie der Mensch. 

\subsection{Trainingsdaten}
Mithilfe der Beispiele des Trainingsdatensatz soll das Model

\end{document}
