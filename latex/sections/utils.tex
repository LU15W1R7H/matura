\newcommand{\mat}[1]{\mathbf{#1}}
\let\oldvec\vec
\renewcommand{\vec}[1]{\mathbf{#1}}

%transpose
\newcommand{\trs}[1]{#1^\mathrm{T}}


% \renewcommand{\vec}[1]{\vv{#1}}

% ----------- TIKZ ---------------


\NewEnviron{defbox}[1]
{
  \centering
  
  \tikzstyle{mybox} = [draw=red, fill=blue!40!bgcolor, very thick, rectangle, rounded corners, inner sep=10pt, inner ysep=20pt]
  \tikzstyle{fancytitle} = [fill=red, text=white]
  \begin{tikzpicture}
    \node [mybox] (box) {%
      \begin{minipage}{1.0\textwidth}
        \BODY
      \end{minipage}
    };
    \node [right,inner xsep=1em,fill=red!75, text=white,outer sep=0pt,text height=2ex,text depth=.5ex] (title) 
    at ([shift={(-1em,0pt)}]box.north west) {#1};
    \fill[red!50!black] (title.north east) -- +(-1em,1em) -- +(-1em,0) -- cycle;
    \fill[red!50!black] (title.south west) -- +(1em,-1em) -- +(1em,0) -- cycle;

  \end{tikzpicture}
}

\NewEnviron{infobox}[1]
{
  
  \tikzstyle{mybox} = [draw=blue, fill=green!20!black, very thick, rectangle, rounded corners, inner sep=10pt, inner ysep=20pt]
  \tikzstyle{fancytitle} = [fill=blue, text=white]
  \begin{tikzpicture}
    \node [mybox] (box) {%
      \begin{minipage}{1.0\textwidth}
        \BODY
      \end{minipage}

    };
    \node[fancytitle,right=10pt] at (box.north west) {#1};

  \end{tikzpicture}
}



%%% Local Variables:
%%% mode: latex
%%% TeX-master: "../main"
%%% End:
